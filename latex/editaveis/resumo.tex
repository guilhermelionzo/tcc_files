\begin{resumo}
	
Pesquisadores da Universidade de Brasília (UnB) começaram a estudar a viabilidade de uma missão CubeSat 3U, para demonstrador tecnológico. Alguns estudos já começaram a ser realizados, com o intuito de oferecer soluções para essa futura missão. O presente projeto de pesquisa visa a construção de um Computador de Bordo (OBC) para esse CubeSat. Foi utilizado a metodologia de \textit{Co-Design}, que permitiu o desenvolvimento do \textit{hardware} e \textit{software} simultaneamente. Durante a concepção do projeto teórico, foi escolhido o microcontrolador e outros dispositivos para compor o \textit{hardware} do OBC. Já para o \textit{software} embarcado foi definido como sistema operacional o FreeRTOS. Durante os testes em protoboard foi possível verificar: o consumo do microcontrolador; modos de operação do software embarcado; a aquisição e armazenamento de dados; etc. Foi possível concluir que o uso do TI MSP432 é uma ótima opção para cenários de baixo consumo e performance intermediaria. O uso do FreeRTOS como sistema operacional de tempo-real para sistemas com pouca memória, bem como o uso de Utilização do Watchdog a nível de software, foi ratificada. Alguns requisitos estabelecidos no projeto não foram cumpridos devido à complexidade do projeto. Os pontos que não foram desenvolvidos ou aprofundados durante a pesquisa foram levantados e estão presentes na última seção.

 \vspace{\onelineskip}
    
 \noindent
 \textbf{Palavras-chave}: OBC. CubeSat. Computador de Bordo. Nano Satélites.

\end{resumo}
