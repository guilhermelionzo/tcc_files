\begin{resumo}[Abstract]
 \begin{otherlanguage*}{english}
 	
Researchers from the University of Brasília (UnB) are studying the feasibility of a CubeSat 3U mission, as a technology demonstrator. Some studies are already being carried out, in order to offer solutions for this future mission. The present research is aimed at the construction of an Onboard Computer (OBC) for this future mission. During the development of the OBC, it was used the co-design methodology, which allowed for the development of hardware and software at the same time. During the design of the theoretical project, it was chosen the microcontroller and another devices to compose the OBC's \textit {hardware}. For the embedded software, the FreeRTOS operating system was defined as the operating system. During the protoboard test, it was possible to verify: the consumption of the microcontroller; modes of operation of the embedded software; the acquisition and data storage; etc. It was concluded that the use of the TI MSP432 is a great choice for low-power and intermediate performance scenarios. The use of FreeRTOS as a real-time operating system for low memory systems, as well as the use of watchdog utilization at software level has been ratified. Some requirements established at the beginning were not fulfilled due to the complexity of the project. Points that were not developed during the search were raised and are present in the last section.
   \vspace{\onelineskip}
 
   \noindent 
   \textbf{Key-words}: OBC. CubeSat. On-Board Computer. Nanosatellites.
 \end{otherlanguage*}
\end{resumo}
